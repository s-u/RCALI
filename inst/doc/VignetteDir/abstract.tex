\begin{abstract}

\href{http://w3.jouy.inra.fr/unites/miaj/public/logiciels/RCALI/}{RCALI}
\footnote{\href{http://w3.jouy.inra.fr/unites/miaj/public/logiciels/RCALI}{http://w3.jouy.inra.fr/unites/miaj/public/logiciels/RCALI}}  is a \href{http://www.r-project.org/}{R}~\cite{R:2011} package that makes the
interface between  the \href{http://w3.jouy.inra.fr/unites/miaj/public/logiciels/califlopp/}{CaliFloPP}\footnote{\href{http://w3.jouy.inra.fr/unites/miaj/public/logiciels/califlopp}{http://w3.jouy.inra.fr/unites/miaj/public/logiciels/califlopp}} and R.

\verb+CaliFloPP+ is a software that calculates flows of particles between
pairs of polygons, when given a so-called individual dispersal
function. The individual dispersal function describes the particle
dispersion between pairs of points, and \verb+CaliFloPP+ deduces the total
flows between pairs of polygons. This integration problem is solved by
reducing the dimension of the integral and by using algorithms from
computational geometry (see~\cite{Bouvier:2009}).

In addition, \verb+RCALI+ allows to take into account the
angle of the current point with the horizontal and so, define
anisotropic  dispersal
functions.

This manual first describes the methods implemented in
\verb+CaliFloPP+, then illustrates how to use it through  \verb+RCALI+,
 and last, gives some
hints to customize the package.
\vspace{5mm}

\centerline{\textbf{Résumé}}
\vspace{5mm}
\href{http://w3.jouy.inra.fr/unites/miaj/public/logiciels/RCALI/}{RCALI}$^1$ est un paquetage \href{http://www.r-project.org/}{R}~\cite{R:2011}
qui interface le logiciel
\href{http://w3.jouy.inra.fr/unites/miaj/public/logiciels/califlopp/welcome_french.html}{CaliFloPP}$^2$
à R.

Le logiciel \verb+ CaliFloPP+ estime 
des flux de particules entre paires de polygones: 
à partir d'une fonction de dispersion dite
individuelle, c'est-à-dire décrivant la dispersion des particules de
point à point, il calcule les flux totaux émis d'un polygone
à un autre. Ce problème d'intégration est résolu en 
réduisant la dimension de l'intégrale et en utilisant des
algorithmes de géométrie algorithmique  (voir~\cite{Bouvier:2009}).

\verb+RCALI+ permet en outre de prendre en compte l'angle du point
courant avec l'horizontale, et ainsi, de définir des fonctions
de dispersion anisotropiques.


Cette notice décrit  les méthodes implémentées dans
\verb+CaliFloPP+, illustre comment l'utiliser via \verb+RCALI+,
et comment adapter le paquetage.
\end{abstract}
